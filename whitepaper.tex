\documentclass{article}
\usepackage[utf8]{inputenc}

\title{The Real World Transport Protocol}
\author{Evan Conrad}
\date{April 2022}

\begin{document}

\maketitle

\section{Introduction}

The Real World Transport Protocol (RWTP) defines a way for two parties, who may not trust each other, to securely exchange physical goods via a decentralized ledger, such as a blockchain. Let the first party be $Seller$, and the second party be $Buyer$. 

To create a sell order, the $Seller$ generates a new public-private key-pair ($Item$) to represent the item they are seller. Then, the $Seller$ transfers some amount of a $Currency$\footnote{For example, Ether, DAI, USDC, SOL} into an escrow ($Stake_{seller}$), to be used as collateral. A potential $Buyer$ submits a request to purchase by transferring the payment into an escrow (referred to as the $Payment$) and additionally transfers funds into another escrow ($Stake_{buyer}$). The $Buyer$ also includes the public key $Buyer_{pu}$ of their own public-private keypair. If the $Seller$ accepts the purchase request, then no other potential $Buyer$ can make a request and neither the $Seller$ or $Buyer$ can, at this point, withdraw from the escrows. 

To deliver the item, the $Seller$ generates a public-key pair to represent the $Item$. They encrypt the $Item$ private key with the public key of the buyer to create a ShippingKey.

\begin{equation}
    ShippingKey = Encrypt_{Buyer_{pu}}(Item_{pk})
\end{equation} 

Because the $ShippingKey$ is encrypted, it can be added to the package publicly such as on a QR-code or an NFC-capable device. When the $Buyer$ receives the package, they decrypt the $ShippingKey$ with their own private keypair, to get the Item's private key ($Item_{pk}$). Finally, they sign a message with the $Item_{pk}$ and publish it to the immutable ledger. At this point, the $Payment$ moves to the $Seller$, the $Stake_{buyer}$ is returned to the buyer, and the $Stake_{seller}$ is returned to the seller. However, if after some time period\footnote{Or, perhaps use-defined rules}, the $Item_{pk}$ have not signed a message, the $Currency$ in both party's escrow is permanently destroyed.

\section{Game Theory}

Let $S_{yes}$ be the outcome where the $Seller$ delivers the item and $S_{no}$ be the outcome where the $Seller$ does not to deliver the item. Let $B_{yes}$ be the outcome where the $Buyer$ signs that they received the item. Let $B_{no}$ be the outcome where the buyer signs that they did not receive the item. Let $I$ be the market value of the item at the point of exchange. Let $P$ be the payment for the item. Because an exchange has occurred, we assume that the value of the item and the payment are equal.\footnote{This may not necessarily be the case! For example, the price of the item may go up after purchase, but before delivery.}

We can then define a pay-off matrix that shows the results of different outcomes

\begin{equation}
\begin{tabular}{  c | c | c  }
       & $S_{yes}$ & $S_{no}$ \\
       \hline
 $B_{yes}$ & $(0, 0)$ & $(-Stake_{buyer}, -Stake_{seller}+P)$ \\ 
 $B_{no}$ & $(-Stake_{buyer}+I, -Stake_{seller}))$ & $(-Stake_{buyer}, -Stake_{seller})$ \\  
\end{tabular}
\end{equation}

$B_{yes}$ and $S_{yes}$ is the saddle point for the $Buyer$ if the following applies.

\begin{equation}
    Stake_{buyer} > I
\end{equation}

$B_{yes}$ and $S_{yes}$ is the saddle point for the $Seller$ if the following applies.

\begin{equation}
    Stake_{seller} > P
\end{equation}

In a finite game, RWTP works if both parties over-collateralized. However, most real-world trade is not a finite game. Therefore both buyers and sellers may be able to reduce the stake required depending on their level of trust after repeated purchases. 

If the $Buyer$ and $Seller$ have perfect trust in each other, then the stake for both sides can be 0. If the $Buyer$ and $Seller$ have no trust for each other, then the stake for both sides must be equal to $I$ and $P$.

\section{Motivation}

RWTP lets you program the real world. 

It allows for previously impracticable use cases of decentralized ledgers, programmatic supply chains, automated companies, decentralized futures over real-world goods, decentralized competitors to Amazon and other e-commerce sites, and potentially more.

\end{document}
